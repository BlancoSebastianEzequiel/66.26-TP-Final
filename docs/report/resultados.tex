\subsection{Multiplicacion por bloques}

    \def\model{ByBlocks}
    \def\analisisAmdahl{
    Podemos observar que el speed up teorico tiende al
    maximo speed up mientras que el real encuentra un maximo luego de los 4
    thread aproximadamente ya que la cantidad maxima de core de la computadora
    donde se corrio el programa es de 4
    }
    \def\analisisGustafson{
    Podemos ver que estos resultados demuestran que la sección serie del
    problema se mantiene casi constante respecto de la sección paralela
    que varía en forma ascendente con el tamaño de los datos de entrada.
    Así, la fracción secuencial representa menos al tiempo total en la medida
    que la carga de trabajo aumenta
    }
    \paragraph{Salida Amdahl}

    \def\versiondir{graphs_v0/}

    \def\text{Salida de los tiempos en serie y paralelo}
    \def\path{\versiondir\model_amdahl_output_table.png}
    \def\scale{.6}
    \begin{figure}[!htbp]
    \begin{adjustbox}{addcode={
        \begin{minipage}{\width}}{
            \caption{\text}
        \end{minipage}},rotate=360,center}
        \includegraphics[scale=\scale]{\path}
    \end{adjustbox}
\end{figure}
\FloatBarrier

    De acuerdo a estos datos podemos calcular el speed up maximo, real y teórico.

    \def\text{Speed up real, teorico y maximo segun la cantidad de threads}
    \def\path{\versiondir\model_amdahl_speed_up_table.png}
    \def\scale{.6}
    \begin{figure}[!htbp]
    \begin{adjustbox}{addcode={
        \begin{minipage}{\width}}{
            \caption{\text}
        \end{minipage}},rotate=360,center}
        \includegraphics[scale=\scale]{\path}
    \end{adjustbox}
\end{figure}
\FloatBarrier

    \def\text{Grafico}
    \def\path{\versiondir\model_amdahl_speed_up.png}
    \def\scale{.6}
    \begin{figure}[!htbp]
    \begin{adjustbox}{addcode={
        \begin{minipage}{\width}}{
            \caption{\text}
        \end{minipage}},rotate=360,center}
        \includegraphics[scale=\scale]{\path}
    \end{adjustbox}
\end{figure}
\FloatBarrier

    \analisisAmdahl

    \newpage

\paragraph{Salida Gustafson}

    \def\text{Salida de los tiempos en serie y paralelo con el error}
    \def\path{\versiondir\model_gustafson_output_table.png}
    \def\scale{.6}
    \begin{figure}[!htbp]
    \begin{adjustbox}{addcode={
        \begin{minipage}{\width}}{
            \caption{\text}
        \end{minipage}},rotate=360,center}
        \includegraphics[scale=\scale]{\path}
    \end{adjustbox}
\end{figure}
\FloatBarrier

    \analisisGustafson

    \def\text{Tiempo paralelo y serie en funcion de la dimension de las matrices de entrada}
    \def\path{\versiondir\model_gustafson_exec_time.png}
    \def\scale{.6}
    \begin{figure}[!htbp]
    \begin{adjustbox}{addcode={
        \begin{minipage}{\width}}{
            \caption{\text}
        \end{minipage}},rotate=360,center}
        \includegraphics[scale=\scale]{\path}
    \end{adjustbox}
\end{figure}
\FloatBarrier

    \clearpage
    \newpage

    Luego a partir de estos datos podemos calcular el speed up y obtuvimos lo
    siguiente:

    \def\text{Tabla de valores del speed up}
    \def\path{\versiondir\model_gustafson_speed_up_table.png}
    \def\scale{.6}
    \begin{figure}[!htbp]
    \begin{adjustbox}{addcode={
        \begin{minipage}{\width}}{
            \caption{\text}
        \end{minipage}},rotate=360,center}
        \includegraphics[scale=\scale]{\path}
    \end{adjustbox}
\end{figure}
\FloatBarrier

    \def\text{Grafico del speed up}
    \def\path{\versiondir\model_gustafson_speed_up.png}
    \def\scale{.6}
    \begin{figure}[!htbp]
    \begin{adjustbox}{addcode={
        \begin{minipage}{\width}}{
            \caption{\text}
        \end{minipage}},rotate=360,center}
        \includegraphics[scale=\scale]{\path}
    \end{adjustbox}
\end{figure}
\FloatBarrier

    \newpage
    \clearpage

\subsection{Multiplicacion elemento por fila}

    \def\model{ElementByRowBlock}
    \def\analisisAmdahl{
    Podemos observar que el speed up teorico tiende al
    maximo speed up mientras que el real encuentra un maximo luego de los 4
    thread aproximadamente ya que la cantidad maxima de core de la computadora
    donde se corrio el programa es de 4
    }
    \def\analisisGustafson{
    Podemos ver que estos resultados demuestran que la sección serie del
    problema se mantiene casi constante respecto de la sección paralela
    que varía en forma ascendente con el tamaño de los datos de entrada.
    Así, la fracción secuencial representa menos al tiempo total en la medida
    que la carga de trabajo aumenta
    }
    \paragraph{Salida Amdahl}

    \def\versiondir{graphs_v0/}

    \def\text{Salida de los tiempos en serie y paralelo}
    \def\path{\versiondir\model_amdahl_output_table.png}
    \def\scale{.6}
    \begin{figure}[!htbp]
    \begin{adjustbox}{addcode={
        \begin{minipage}{\width}}{
            \caption{\text}
        \end{minipage}},rotate=360,center}
        \includegraphics[scale=\scale]{\path}
    \end{adjustbox}
\end{figure}
\FloatBarrier

    De acuerdo a estos datos podemos calcular el speed up maximo, real y teórico.

    \def\text{Speed up real, teorico y maximo segun la cantidad de threads}
    \def\path{\versiondir\model_amdahl_speed_up_table.png}
    \def\scale{.6}
    \begin{figure}[!htbp]
    \begin{adjustbox}{addcode={
        \begin{minipage}{\width}}{
            \caption{\text}
        \end{minipage}},rotate=360,center}
        \includegraphics[scale=\scale]{\path}
    \end{adjustbox}
\end{figure}
\FloatBarrier

    \def\text{Grafico}
    \def\path{\versiondir\model_amdahl_speed_up.png}
    \def\scale{.6}
    \begin{figure}[!htbp]
    \begin{adjustbox}{addcode={
        \begin{minipage}{\width}}{
            \caption{\text}
        \end{minipage}},rotate=360,center}
        \includegraphics[scale=\scale]{\path}
    \end{adjustbox}
\end{figure}
\FloatBarrier

    \analisisAmdahl

    \newpage

\paragraph{Salida Gustafson}

    \def\text{Salida de los tiempos en serie y paralelo con el error}
    \def\path{\versiondir\model_gustafson_output_table.png}
    \def\scale{.6}
    \begin{figure}[!htbp]
    \begin{adjustbox}{addcode={
        \begin{minipage}{\width}}{
            \caption{\text}
        \end{minipage}},rotate=360,center}
        \includegraphics[scale=\scale]{\path}
    \end{adjustbox}
\end{figure}
\FloatBarrier

    \analisisGustafson

    \def\text{Tiempo paralelo y serie en funcion de la dimension de las matrices de entrada}
    \def\path{\versiondir\model_gustafson_exec_time.png}
    \def\scale{.6}
    \begin{figure}[!htbp]
    \begin{adjustbox}{addcode={
        \begin{minipage}{\width}}{
            \caption{\text}
        \end{minipage}},rotate=360,center}
        \includegraphics[scale=\scale]{\path}
    \end{adjustbox}
\end{figure}
\FloatBarrier

    \clearpage
    \newpage

    Luego a partir de estos datos podemos calcular el speed up y obtuvimos lo
    siguiente:

    \def\text{Tabla de valores del speed up}
    \def\path{\versiondir\model_gustafson_speed_up_table.png}
    \def\scale{.6}
    \begin{figure}[!htbp]
    \begin{adjustbox}{addcode={
        \begin{minipage}{\width}}{
            \caption{\text}
        \end{minipage}},rotate=360,center}
        \includegraphics[scale=\scale]{\path}
    \end{adjustbox}
\end{figure}
\FloatBarrier

    \def\text{Grafico del speed up}
    \def\path{\versiondir\model_gustafson_speed_up.png}
    \def\scale{.6}
    \begin{figure}[!htbp]
    \begin{adjustbox}{addcode={
        \begin{minipage}{\width}}{
            \caption{\text}
        \end{minipage}},rotate=360,center}
        \includegraphics[scale=\scale]{\path}
    \end{adjustbox}
\end{figure}
\FloatBarrier

    \newpage
    \clearpage

\subsection{Multiplicacion columna por fila}

    \def\model{ColumnByRow}
    \def\analisisAmdahl{
    Podemos observar que el speed up teorico tiende al
    maximo speed up mientras que el real empieza a tender a un maximo menor
    debido a que la cantidad maxima de core de la computadora donde se corrio
    el programa es de 4. Pero podemos ver tambien que hubo variaciones y que el
    speed up real tuvo anomalias y tendencias a crecer a pesar de el maximo de
    cores reales. Esto se debe a que al aumentar la cantidad de thread, la
    seccion serie del problema no se mantuvo constante si no que tuve ciertas
    variaciones equilibrando la caida de perfonmance de la seccion paralela.
    Esta variacion en el tiempo de la parte serie se debe a la forma en que
    quedan organizados los datos una vez procesados por los map workers
    }
    \def\analisisGustafson{
    Podemos ver que estos resultados demuestran que la sección serie del
    problema se mantiene casi constante respecto de la sección paralela
    que varía en forma ascendente con el tamaño de los datos de entrada.
    Así, la fracción secuencial representa menos al tiempo total en la medida
    que la carga de trabajo aumenta
    }
    \paragraph{Salida Amdahl}

    \def\versiondir{graphs_v0/}

    \def\text{Salida de los tiempos en serie y paralelo}
    \def\path{\versiondir\model_amdahl_output_table.png}
    \def\scale{.6}
    \begin{figure}[!htbp]
    \begin{adjustbox}{addcode={
        \begin{minipage}{\width}}{
            \caption{\text}
        \end{minipage}},rotate=360,center}
        \includegraphics[scale=\scale]{\path}
    \end{adjustbox}
\end{figure}
\FloatBarrier

    De acuerdo a estos datos podemos calcular el speed up maximo, real y teórico.

    \def\text{Speed up real, teorico y maximo segun la cantidad de threads}
    \def\path{\versiondir\model_amdahl_speed_up_table.png}
    \def\scale{.6}
    \begin{figure}[!htbp]
    \begin{adjustbox}{addcode={
        \begin{minipage}{\width}}{
            \caption{\text}
        \end{minipage}},rotate=360,center}
        \includegraphics[scale=\scale]{\path}
    \end{adjustbox}
\end{figure}
\FloatBarrier

    \def\text{Grafico}
    \def\path{\versiondir\model_amdahl_speed_up.png}
    \def\scale{.6}
    \begin{figure}[!htbp]
    \begin{adjustbox}{addcode={
        \begin{minipage}{\width}}{
            \caption{\text}
        \end{minipage}},rotate=360,center}
        \includegraphics[scale=\scale]{\path}
    \end{adjustbox}
\end{figure}
\FloatBarrier

    \analisisAmdahl

    \newpage

\paragraph{Salida Gustafson}

    \def\text{Salida de los tiempos en serie y paralelo con el error}
    \def\path{\versiondir\model_gustafson_output_table.png}
    \def\scale{.6}
    \begin{figure}[!htbp]
    \begin{adjustbox}{addcode={
        \begin{minipage}{\width}}{
            \caption{\text}
        \end{minipage}},rotate=360,center}
        \includegraphics[scale=\scale]{\path}
    \end{adjustbox}
\end{figure}
\FloatBarrier

    \analisisGustafson

    \def\text{Tiempo paralelo y serie en funcion de la dimension de las matrices de entrada}
    \def\path{\versiondir\model_gustafson_exec_time.png}
    \def\scale{.6}
    \begin{figure}[!htbp]
    \begin{adjustbox}{addcode={
        \begin{minipage}{\width}}{
            \caption{\text}
        \end{minipage}},rotate=360,center}
        \includegraphics[scale=\scale]{\path}
    \end{adjustbox}
\end{figure}
\FloatBarrier

    \clearpage
    \newpage

    Luego a partir de estos datos podemos calcular el speed up y obtuvimos lo
    siguiente:

    \def\text{Tabla de valores del speed up}
    \def\path{\versiondir\model_gustafson_speed_up_table.png}
    \def\scale{.6}
    \begin{figure}[!htbp]
    \begin{adjustbox}{addcode={
        \begin{minipage}{\width}}{
            \caption{\text}
        \end{minipage}},rotate=360,center}
        \includegraphics[scale=\scale]{\path}
    \end{adjustbox}
\end{figure}
\FloatBarrier

    \def\text{Grafico del speed up}
    \def\path{\versiondir\model_gustafson_speed_up.png}
    \def\scale{.6}
    \begin{figure}[!htbp]
    \begin{adjustbox}{addcode={
        \begin{minipage}{\width}}{
            \caption{\text}
        \end{minipage}},rotate=360,center}
        \includegraphics[scale=\scale]{\path}
    \end{adjustbox}
\end{figure}
\FloatBarrier

    \newpage
    \clearpage

\subsection{Cblas e instrucciones vectorizadas}
    \def\text{Tiempo serie de multiplicacion en segundos}
    \def\path{\versiondir/dgemm.png}
    \def\scale{.6}
    \begin{figure}[!htbp]
    \begin{adjustbox}{addcode={
        \begin{minipage}{\width}}{
            \caption{\text}
        \end{minipage}},rotate=360,center}
        \includegraphics[scale=\scale]{\path}
    \end{adjustbox}
\end{figure}
\FloatBarrier

    Podemos decir que en general, las corridas con mapReduce, en el caso donde se
    uso un solo thread, si sumamos el tiempo paralelo y serie (que es todo serie
    ya que se usa un solo thread) nos da en promedio de los tres tipos de
    multiplicacion tardaron alrededor de 3 segundos. En cambio podemos ver que
    usando \code{cblas} tenemos un tiempo de 0.03 segundos y usando las
    instrucciones vectorizadas 0.36 segundos.\\
    De esta manera vemos que \code{cblas} y las instrucciones vectorizadas aprovechan
    mucho mejor el hardware para la realizacion de la misma operacion. Tambien es
    cierto que el map reduce hace uso de un modulo de \code{python} llamado
    \code{pool} que puede ser que sume latencia al momento de dividir el trabajo,
    pero cuando aumentamos mucho el volumen del trabajo se ve en los graficos que
    se aprovecha mejor la paralelizacion.

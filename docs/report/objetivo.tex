Se propone la verificación empírica de la ley de Amdahl  (trabajo constante)
versus la ley de  Gustafson (tiempo constante) aplicada a un problema de
paralelismo, utilizando el modelo de programación MapReduce.\\
En Amdahl se hará una multiplicación de matrices (ambas de \code{NxN}) y se
realizarán las mediciones de tiempo, variando la cantidad de threads involucrados
en el procesamiento.\\
Luego, en Gustafson se realizará las mismas mediciones aumentando la dimensión
de las matrices (aumentando el trabajo) y aumentando la cantidad de threads cada
vez que aumentamos el trabajo.\\
Finalmente, se hará una multiplicación de dos matrices diferentes de \code{NxN}
usando la librería CBLAS e instrucciones de vectoriales (MMX) para el compilador
con sólo un procesador. De esta manera, la idea es comparar el tiempo que tarda
el map-reduce en serie frente a CBLAS y la vectorización.

Se propone la verificación empírica de la ley de Amdahl  (trabajo constante)
versus la ley de  Gustafson (tiempo constante) aplicada a un problema de
paralelismo utilizando el modelo de programación MapReduce.\\
Haremos una multiplicación de matrices (ambas de \code{NxN}) y se realizarán las
mediciones de tiempo variando la cantidad de threads involucrados en el
procesamiento.\\
Luego se realizarán las mismas mediciones manteniendo fija la cantidad de
threads pero variando la dimensión de las matrices.\\
Finalmente se hará una multiplicación de dos matrices diferentes de \code{NxN}
usando la librería CBLAS e instrucción de vectorización (MMX) para el compilador
con solo un procesador. De esta manera la idea es comparar el tiempo que tarda
el map-reduce en serie frente a cblas y la vectorizacion.

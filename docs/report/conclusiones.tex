Podemos decir que obtuvimos resultados inesperados pero se tuvieron que hacer
varias corridas y ajustar ciertos números para entender por que llegamos a
éstos\\
Se coloco un \code{sleep} de medio segundo (que no afectó el cálculo
del tiempo paralelo-serie transcurrido) para evitar que cualquier trabajo que no
sea puramente vinculado a la CPU afecte nuestro porgrama (como por ejemplo I/O) \\\\
Finalmente podemos decir que hay que tener en cuenta que hay otros programas
corriendo en las cuatro CPU y que dependiendo del tamaño de informacion que
manejamos, podemos tener un cuello de botella ya sea por intercambios de memoria
o por exceso de memoria.
Entonces se tuvieron que hacer varias corridas analizando el tráfico de
información mediante el comando \code{gnome-system-monitor} donde filtrando
los procesos y solo viendo los de \code{python} pudimos ver el uso de cada CPU
y gráficos al respecto.

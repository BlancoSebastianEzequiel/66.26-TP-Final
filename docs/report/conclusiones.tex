Podemos decir que obtuvimos resultados esperados pero se tuvieron que hacer
varias corridas y ajustar ciertos numeros para entender por que llegamos a estos
resultados. \\
Primero para que las curvas de speed up de Amdahl nos den bien habia
que tener en cuenta que la dimension de las matrices tenia que ser lo
suficientemente grandes como para tener un becnhmark rasonable, pero tambien hay
un limite superior para el cual no superamos la capacidad de los procesadores.
Luego, una vez mapeados los datos, se aprovecho el uso de los multiprocesadores
para reordenar los datos de manera tal que la parte de \code{reduce} pueda leerlos.
Es decir, cuando los ordenamos lo hacemos dividiendo el trabajo en partes donde
cada una la realiza cada procesador. De esta manera obtenemos una organizacion
tipo arbol donde evitamos un cuello de botella.\\
Y segundo se coloco un \code{sleep} de medio segundo (que no afecto el calculo
del tiempo paralelo-serie transcurrido) para evitar que cualquier trabajo que no
sea puramente vinculado a la CPU afecte nuestro porgrama (como por ejemplo I/O)\\
Finalmente podemos decir que hay que tener en cuenta que hay otros programas
corriendo en las cuatro CPU que tiene la computadora en la cual se probo este
programa y que dependiendo del tamaño de informacion que manejamos, podemos tener
un cuello de botella ya sea por intercambios de memoria o por exceso de memoria.
Entonces se tuvieron que hacer varias corridas analizando el trafico de
informacion mediante el comando \code{gnome-system-monitor} donde filtrando
los procesos y solo viedno los de \code{python} pudimos ver el uso de cada CPU
y graficos al respecto.

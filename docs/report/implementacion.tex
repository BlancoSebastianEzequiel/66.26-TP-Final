\subsection{Explicación del modelo}
    La implementación del MapReduce para resolver el problema esta basado en el
    siguiente esquema:\\

    \def\text{Esquema de un map reduce}
    \def\path{map_reduce_schema.png}
    \def\scale{.6}
    \begin{figure}[!htbp]
    \begin{adjustbox}{addcode={
        \begin{minipage}{\width}}{
            \caption{\text}
        \end{minipage}},rotate=360,center}
        \includegraphics[scale=\scale]{\path}
    \end{adjustbox}
\end{figure}
\FloatBarrier

    En nuestro caso creamos una clase llamada \code{MapReduce} la cual usa una
    librería de \code{python} llamada \code{multiprocessing} en donde usamos el
    módulo \code{pool} el cual ofrece un medio conveniente para paralelizar la
    ejecución de una función a través de múltiples valores de entrada, distribuyendo
    los datos de entrada a través de procesos (paralelismo de datos).\\

    Entonces lo que hicimos fue instanciar un \code{pool} para hacer el map de
    manera que se le pasa como atributo la cantidad de workers en el cual se
    quiere paralelizar el problema.

\subsection{Multiplicación de matrices por bloques}

    \paragraph{Preprocesamiento}

        Sean dos matrices \code{A} de \code{NxN} y \code{B} de \code{NxN} las
        dividimos en \code{(N/2)x(N/2)} bloques cada una. Luego generamos una
        lista de tuplas donde cada una tiene la posición \code{(r, c)} de un bloque
        de la matriz \code{A}, tiene el bloque en custión \code{a\_block\_rc},
        y la fila número \code{c} de bloques de la matriz \code{B}, quedando
        con este formato: \\
        \code{(r, c, a\_block\_rc, b\_block\_c)}

    \paragraph{Mapeo}

        Recibimos la posición \code{r, c} del bloque \code{a}, el bloque
        \code{a} y una lista de bloques \code{b} que es la fila \code{c} de
        bloques en la matriz B.\\
        Entonces multiplicamos el bloque \code{a} por cada bloque de la lista de
        bloques \code{b} y guardamos en un vector una tupla con una clave
        \code{r, c\_b} donde \code{c\_b} es el indice en la lista de
        bloques \code{b} y como valor guardamos la multiplicación. Por cada
        multiplicación, agregamos una de estas tuplas al vector de salida para
        luego devolver éste.

    \paragraph{Reducción}

        Recibimos la posición de un bloque de salida y una lista de
        multiplicaciones parciales de bloques. Se suman estas multiplicaciones
        parciales y se devuelve un vector con los valores resultantes de la
        multiplicación. Pero por cada valor se calcula la posición de salida del
        mismo en la matriz resultante y nos deshacemos de la posición de los
        bloques

\subsection{Multiplicación de matrices de elemento por fila}

    \paragraph{Preprocesamiento}

        Sean dos matrices \code{A} de \code{NxN} y \code{B} de \code{NxN}
        generamos una lista de tuplas a partir de las dos matrices.
        Se itera por cada elemento (\code{a\_ij}) de la matriz \code{A} y se
        guarda en cada tupla el número de fila \code{i} del elemento
        \code{a\_ij}, el elemento \code{a\_ij} y la fila \code{j} de la
        matriz \code{B}. Quedando cada tupla de la siguiente manera:\\
        \code{(i, a\_ij, B[j])}

    \paragraph{Mapeo}

        De esta manera, en la función map, obtenemos partes de esta lista de
        tuplas y devolvemos un par clave, valor donde la clave es la posición
        de salida de la matriz resultante \code{(i, j)} y el valor es la
        multiplicación del elemento \code{a\_ij} contra cada elemento de la
        fila \code{j} de la matriz B

    \paragraph{Reducción}

        Obtenemos una posición de salida y una lista de valores que resultaron
        de la multiplicación que se hizo en el map. Entonces se suman las
        multiplicaciones parciales y se obtiene el valor en la posición de salida
        de la matriz resultante

\subsection{Multiplicación de matrices de columna por fila}

    \paragraph{Preprocesamiento}

        Sean dos matrices \code{A} de \code{NxN} y \code{B} de \code{NxN}
        generamos una lista de tuplas a partir de las dos matrices.
        Se guarda en cada tupla la columna \code{i} de la matriz \code{A} y
        la fila \code{i} de la matriz \code{B}. Quedando cada tupla de la
        siguiente manera:\\
        \code{(A[:][i], B[i])}

    \paragraph{Mapeo}
        Recibimos una columna de la matriz A y una fila de la matriz B y por cada
        elemento de la columna \code{elem\_a} lo multiplicamos por cada elemento
        de la fila \code{elem\_b} obteniendo una matriz parcial de la
        multiplicación. Por cada multiplicación guardamos en un vector una tupla
        con un par clave valor donde la clave es la posición de salida de la matriz
        resultante y el valor es la multiplicación anteriormente mencionada.
        Finalmente se devuelve el vector de tuplas.

    \paragraph{Reducción}

        Se recibe la posición de salida de la matriz resultante y una lista de
        multiplicaciones parciales. Entonces se suman éstas y se devuelve la
        posición de salida y la suma.

\subsection{MMX}
    Se escribió un código en \code{c}, donde se multiplican dos matrices
    por bloques. Al momento de hacer la multiplicación parcial de elementos, donde
    cada uno es un bloque de la matriz, se usa una intrucción llamada
    \code{\#pragma vector always} para decirle al  compilador que la siguiente
    fracción de código será vectorizada.\\
    Este código fue dessarrollado en el archivo \code{blocked\_dgemm\_sse.c} que
    se encuentra en el anexo.

\subsection{Cblas}
    Se uso una función de la librería \code{cblas.h} de \code{c}. Esta función
    cuya firma es:
    \lstset{language=C, numbers=left, tabsize=2, escapeinside=||}
    \begin{lstlisting}
    void cblas_dgemm(
            CBLAS_LAYOUT layout,
            CBLAS_TRANSPOSE TransA,
            CBLAS_TRANSPOSE TransB,
            const int M, const int N,
            const int K,
            const double alpha,
            const double *A,
            const int lda,
            const double *B,
            const int ldb,
            const double beta,
            double *C,
            const int ldc);
    \end{lstlisting}
    Ésta llamada a la rutina \code{cblas\_dgemm} multiplica dos matrices:
    \begin{lstlisting}
        cblas_dgemm(CblasRowMajor, CblasNoTrans, CblasNoTrans, m, n, k, alpha, A, k, B, n, beta, C, n);
    \end{lstlisting}
    Los argumentos proporcionan opciones sobre cómo Intel MKL realiza la
    operación. En este caso: \\
    \begin{itemize}
        \item \underline{\textbf{CblasRowMajor:}} Indica que las matrices se
        almacenan en el orden mayor de la fila, con los elementos de cada fila
        de la matriz almacenados de forma contigua.
        \item \underline{\textbf{CblasNoTrans:}} Tipo de enumeración que indica
        que las matrices A y B no deben ser transpuestas o conjugadas antes de
        la multiplicación.
        \item \underline{\textbf{m, n, k:}} Enteros que indican el tamaño de
        las matrices:
        \begin{itemize}
            \item \underline{\textbf{A:}} m filas por k columnas
            \item \underline{\textbf{B:}} k filas por n columnas
            \item \underline{\textbf{C:}} m filas por n columnas
        \end{itemize}
        \item \underline{\textbf{alpha:}} Valor real utilizado para escalar el
        producto de las matrices A y B.
        \item \underline{\textbf{A:}} Arreglo utilizado para almacenar la matriz A.
        \item \underline{\textbf{k:}} Dimensión inicial de la matriz A, o el
        número de elementos entre filas sucesivas (para el almacenamiento
        principal de fila) en la memoria. En el caso de este ejercicio,
        la dimensión principal es la misma que la cantidad de columnas.
        \item \underline{\textbf{B:}} Arreglo utilizado para almacenar la matriz B.
        \item \underline{\textbf{n:}} Dimensión inicial de la matriz B, o el
        número de elementos entre filas sucesivas (para el almacenamiento
        principal de fila) en la memoria. En el caso de este ejercicio, la
        dimensión principal es la misma que la cantidad de columnas.
        \item \underline{\textbf{beta:}} Valor real utilizado para escalar la matriz C.
        \item \underline{\textbf{C:}} Arreglo utilizado para almacenar la matriz C.
        \item \underline{\textbf{n:}} Dimensión inicial de la matriz C, o el
        número de elementos entre filas sucesivas (para el almacenamiento
        principal de fila) en la memoria. En el caso de este ejercicio, la
        dimensión principal es la misma que la cantidad de columnas.
    \end{itemize}


\subsection{Forma de ejecución}
    Para el caso de Amdahl multiplicamos dos matrices de \code{100x100} y cada
    una de estas multiplicaciones la realizamos para \code{1}, \code{2},
    \code{3}, \code{4}, \code{8}, \code{16}, \code{32}, \code{64} y \code{128}
    threads.

    \hfill \break
    Para el caso de Gustafson se usan se multiplican dos matrices de
    \code{100x100} con 1 thread, dos matrices de \code{126x126} con 2 threads y
    dos matrices de \code{158x158} con 4 threads.

    \hfill \break
    Luego para el caso de \code{cblas} y de instrucciones vectoriales (MMX)
    se usa un thread multiplicando dos matrices de \code{400x400}

    \hfill \break
    Para poder probar este trabajo se debe clonar el repositorio (el link esta
    en la carátula) y abrir una terminal en el \code{root} del mismo.

    \hfill \break
    Para compilar \code{cblas}, las instrucciones vectoriales, y las pruebas
    del módulo \code{pool} que están en lenguaje c se debe ejecutar: \\
    \lstinline[columns=fixed]{$ make}.

    \hfill \break
    Para realizar el cálculo de  \code{cblas} y las instrucciones vectoriales
    que están en lenguaje c se debe ejecutar: \\
    \lstinline[columns=fixed]{$ make run_code}.

    \hfill \break
    Para realizar las pruebas del módulo \code{pool} frente a \code{c} y
    \code{python} se debe ejecutar: \\
    \lstinline[columns=fixed]{$ make run_test_pool}.

    \hfill \break
    Para realizar el cálculo de map-reduce se debe ejecutar: \\
    \lstinline[columns=fixed]{$ sh scripts/run.sh}.

    \hfill \break
    Luego para generar los gráficos que vemos en el informe se debe
    ejecutar: \\
    \lstinline[columns=fixed]{$ sh scripts/generate_output_data.sh}

    \hfill \break
    Y finalmente para generar el informe debemos ejecutar: \\
    \lstinline[columns=fixed]{$ sh scripts/make_report.sh}

    \hfill \break
    También hay un script que corre estos últimos tres comandos en un solo script:\\
    \lstinline[columns=fixed]{$ sh scripts/run_all.sh}

\subsection{Datos sobre la computadora que se utilizó}
    El equipo sobre el que se realizarán las mediciones es una laptop con un
    procesador Intel core I7 que posee 4 núcleos a 2.7 GHz, es decir, soporta
    hasta 4 threads en paralelo, con 16 Gb de memoria y corriendo sobre un
    sistema Linux.\\
    Para averiguar estos datos en linux se ejecutaron los siguientes comandos:\\
    \begin{itemize}
        \item \underline{\textbf{Cantidad de cores:}} \lstinline[columns=fixed]{$ grep -c processor /proc/cpuinfo}
        \item \underline{\textbf{Velocidad de reloj:}} \lstinline[columns=fixed]{$ lscpu | grep GHz}
        \item \underline{\textbf{Memoria RAM:}} \lstinline[columns=fixed]{$ free -g}
    \end{itemize}
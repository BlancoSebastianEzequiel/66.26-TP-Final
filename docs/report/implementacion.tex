\subsection{Explicacion del modelo}
    La implementación del MapReduce para resolver el problema esta basado en el
    siguiente esquema:\\

    \def\text{Esquema de un map reduce}
    \def\path{map_reduce_schema.png}
    \def\scale{.6}
    \begin{figure}[!htbp]
    \begin{adjustbox}{addcode={
        \begin{minipage}{\width}}{
            \caption{\text}
        \end{minipage}},rotate=360,center}
        \includegraphics[scale=\scale]{\path}
    \end{adjustbox}
\end{figure}
\FloatBarrier

    En nuestro caso creamos una clase llamada \code{MapReduce} la cual usa una
    libreria de \code{python} llamada \code{multiprocessing} en donde usamos el
    modulo \code{pool} el cual ofrece un medio conveniente para paralelizar la
    ejecución de una función a través de múltiples valores de entrada, distribuyendo
    los datos de entrada a través de procesos (paralelismo de datos).\\

    Entonces lo que hicimos fue instanciar dos \code{pool}, uno para hacer el map y
    el otro para el reduce de manera que el primero se le pasa como atributo la
    cantidad de worker en el cual se quiere paralelizar el problema y el segundo
    solo se usa uno de manera tal que la fase de reduce sea la serie.

\subsection{Multiplicacion de matrices por bloques}

    \paragraph{Preprocesamiento}

        Sean dos matrices \code{A} de \code{NxN} y \code{B} de \code{NxN} las
        dividimos en \code{(N/2)x(N/2)} bloques cada una. Luego generamos una
        lista de tuplas donde cada una tiene la posicion \code{(r, c)} de un bloque
        de la matriz \code{A}, tiene el bloque en custion \code{a\_block\_rc},
        y la fila numero \code{c} de bloques de la matriz \code{B}, quedando
        con este formato: \\
        \code{(r, c, a\_block\_rc, b\_block\_c)}

    \paragraph{Mapeo}

        Recibimos la posicion \code{r}, \code{c} del bloque \code{a}, el bloque
        \code{a} y una lista de bloques \code{b} que es la fila \code{c} de
        bloques en la matriz B.\\
        Entonces multiplicamos el bloque \code{a} por cada bloque de la lista de
        bloques \code{b} y guardamos en un vector una tupla con una clave
        \code{r}, \code{c\_b} donde \code{c\_b} es el indice en la lista de
        bloques \code{b} y como valor guardamos la multiplicacion. Por cada
        multiplicacion, agregamos una de estas tuplas al vector de salida para
        luego devolver este.

    \paragraph{Reduccion}

        Recibimos la posicion de un bloque de salida y una lista de
        multiplicaciones parciales de bloques. Se suman estas multiplicaciones
        parciales y se devuelve un vector con los valores resultantes del la
        multiplicacion. Pero por cada valor se calcula la posicion de salida del
        mismo en la matriz resultante y nos deshacemos de la posicion de los
        bloques

\subsection{Multiplicacion de matrices de elemento por fila}

    \paragraph{Preprocesamiento}

        Sean dos matrices \code{A} de \code{NxN} y \code{B} de \code{NxN}
        generamos una lista de tuplas a partir de las dos matrices.
        Se itera por cada elemento (\code{a\_ij}) de la matriz \code{A} y se
        guarda en cada tupla el numero de fila \code{i} del elemento
        \code{a\_ij}, el elemento \code{a\_ij} y la fila \code{j} de la
        matriz \code{B}. Quedando cada tupla de la siguiente manera:\\
        \code{(i, a\_ij, B[j])}

    \paragraph{Mapeo}

        De esta manera, en la funcion map, obtenemos partes de esta lista de
        tuplas y devolvemos un par clave, valor donde la clave es la posicion
        de salida de la matriz resultante \code{(i, j)} y el valor es la
        multiplicacion del elemento \code{a\_ij} contra cada elemento de la
        fila \code{j} de la matriz B

    \paragraph{Reduccion}

        Obtenemos una posicion de salida y una lista de valores que resultaron
        de la multiplicacion que se hizo en el map. Entonces se suman las
        multiplicaciones parciales y se obtiene el valor en la posicion de salida
        de la matriz resultante

\subsection{Multiplicacion de matrices de columna por fila}

    \paragraph{Preprocesamiento}

        Sean dos matrices \code{A} de \code{NxN} y \code{B} de \code{NxN}
        generamos una lista de tuplas a partir de las dos matrices.
        Se guarda en cada tupla la columna \code{i} de la matriz \code{A} y
        la fila \code{i} de la matriz \code{B}. Quedando cada tupla de la
        siguiente manera:\\
        \code{(A[:][i], B[i])}

    \paragraph{Mapeo}
        Recibimos una columna de la matriz A y una fila de la matriz B y por cada
        elemento de la columna \code{elem\_a} lo multiplicamos por cada elemento
        de la fila \code{elem\_b} obteniendo una matriz parcial de la
        multiplicacion. Por cada multiplicacion guardamos en un vector una tupla
        con un par clave valor donde la clave es la posicion de salida de la matriz
        resultante y el valor es la multiplicacion anteriormente mencionada.
        Finalmente se devuelve el vector de tuplas.

    \paragraph{Reduccion}

        Se recibe la posicion de salida de la matriz resultante y una lista de
        multiplicaciones parciales. Entonces se suman estas y se devuelve la
        posicion de salida y la suma.



\subsection{Forma de ejecucion}
    Para el caso de Amdahl multiplicamos dos matrices de \code{100x100} y cada
    una de estas multiplicaciones la realizamos para \code{1}, \code{2},
    \code{3}, \code{4}, \code{8}, \code{16}, \code{32}, \code{64} y \code{128}
    threads.

    \hfill \break
    Para el caso de gustafson se usan siempre 4 threads multiplicando dos matrices
    de \code{2x2}, \code{4x4}, \code{16x16}, \code{64x64}, \code{100x100},
    \code{200x200}, y \code{300x300}.

    \hfill \break
    Luego para el caso de \code{cblas} y de instrucciones vectoriales (MMX)
    se usa un thread multiplicando dos matrices de \code{400x400}

    \hfill \break
    Para poder probar este trabajo se debe clonar el repositorio (el link esta
    en la caratula) y abrir una terminal en el \code{root} del mismo.

    \hfill \break
    Para compilar \code{cblas} y las instrucciones vectoriales que estan en
    lenguaje c se debe ejecutar: \\
    \lstinline[columns=fixed]{$ make}.

    \hfill \break
    Para realizar el calculo de  \code{cblas} y las instrucciones vectoriales
    que estan en lenguaje c se debe ejecutar: \\
    \lstinline[columns=fixed]{$ ./app}.

    \hfill \break
    Para realizar el calculo de map-reduce se debe ejecutar: \\
    \lstinline[columns=fixed]{$ sh scripts/run.sh}.

    \hfill \break
    Luego para generar los graficos que vemos en el informe se debe
    ejecutar: \\
    \lstinline[columns=fixed]{$ sh scripts/generate_output_data.sh}

    \hfill \break
    Y finalmente para generar el informe debemos ejecutar: \\
    \lstinline[columns=fixed]{$ sh scripts/make_report.sh}

    \hfill \break
    Tambien hay un script que corre estos ultimos tres comandos en un solo script:\\
    \lstinline[columns=fixed]{$ sh scripts/run_all.sh}

\subsection{Datos sobre la computadora que se utilizó}
    El equipo sobre el que se realizarán las mediciones es una laptop con un
    procesador Intel core I7 que posee 4 nucleos a 2.7 Ghz, es decir, soporta
    hasta 4 threads en paralelo, con 16 Gb de memoria y corriendo sobre un
    sistema Linux.\\
    Para averiguar estos datos en linux se ejecutaron los siguientes comandos:\\
    \begin{itemize}
        \item \underline{\textbf{Cantidad de cores:}} \lstinline[columns=fixed]{$ grep -c processor /proc/cpuinfo}
        \item \underline{\textbf{Velocidad de reloj:}} \lstinline[columns=fixed]{$ lscpu | grep GHz}
        \item \underline{\textbf{Memoria RAM:}} \lstinline[columns=fixed]{$ free -g}
    \end{itemize}
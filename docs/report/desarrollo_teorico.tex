\subsection{Speed up}

Es la mejora en la velocidad de ejecución de una tarea ejecutada en dos
arquitecturas similares con diferentes recursos.\\
La noción de speedup fue establecida por la ley de Amdahl, que estaba
dirigida particularmente a la computación paralela. Sin embargo, la speedup se
puede usar más generalmente para mostrar el efecto en el rendimiento después de
cualquier mejora en los recursos.\\
De forma genérica se define como:
\begin{equation}
    speed\_up = \dfrac{Rendimiento\_con\_mejora}{Rendimiento\_sin\_mejora}
\end{equation}
En el caso de mejoras aplicadas a los tiempo de ejecución de una tarea:
\begin{equation}
    speed\_up = \dfrac{T\_ejecucion\_sin\_mejora}{T\_ejecucion\_con\_mejora}
\end{equation}

\subsection{Ley de Amdahl}

Utilizada para averiguar la mejora máxima de un sistema de información cuando
solo una parte de éste es mejorado.\\
Establece que la mejora obtenida en el rendimiento de un sistema debido a la
alteración de uno de sus componentes está limitada por la fracción de tiempo
que se utiliza dicho componente.\\

Suponiendo que nuestro algoritmo se divide en una parte secuencial \code{s}
u una parte paralelizable \code{p} y siendo \code{N} la cantidad de threads,
entonces podemos decir que:
\begin{equation}
    speed\_up = \dfrac{s+p}{s+\dfrac{p}{N}}
\end{equation}

Amdahl establece un límite superior al speedup que puede obtenerse al
introducir una mejora en un determinado algoritmo. Este límite superior está
determinado por la porción de la tarea sobre la que se aplique la mejora.
Entonces si tomamos la ecuacion anterior y calculamos el limite de la misma con
\code{N} tendiendo a infinito tenemos:
\begin{equation}
    speed\_up = \dfrac{1}{s+\dfrac{p}{s}}
\end{equation}
\newpage

\subsection{Ley de Gustafson}

Establece que cualquier problema suficientemente grande puede ser eficientemente
paralelizado. La ley de Gustafson está muy ligada a la ley de Amdahl, que pone
límite a la mejora que se puede obtener gracias a la paralelización, dado un
conjunto de datos de tamaño fijo, ofreciendo así una visión pesimista del
procesamiento paralelo. Por el contrario la ley de Gustafson  propone realizar
mas trabajo con la misma cantidad de recursos, de esta manera aprovecho la
paralelizacion para calcular mas cosas.\\

Entonces siendo \code{s} el tiempo de la ejecucucion de la seccion serie, siendo
\code{p} el tiempo de la ejecucion de la seccion paralela y siendo \code{N} la
cantidad de procesadores podemos calcular el speed up como:
\begin{equation}
    speed\_up = \dfrac{s+p*N}{s+p}
\end{equation}

\subsection{Map-reduce}

MapReduce es una técnica de procesamiento y un programa modelo de computación
distribuida. El algoritmo MapReduce contiene dos tareas importantes.\\\\
Map toma un conjunto de datos y se convierte en otro conjunto de datos, en el
que los elementos se dividen en tuplas (pares: clave, valor).\\

Reduce toma la salida de un mapa como entrada y combina los datos tuplas en un
conjunto más pequeño de tuplas.\\

La principal ventaja de MapReduce es que es fácil de escalar procesamiento de
datos en múltiples nodos.\\

De acuerdo a este modelo, basado en la programación funcional, la tarea del
usuario consiste en la definición de una función map y una función reduce y
definidas estas funciones, el procesamiento es fácilmente paralelizable, ya sea
en una sola máquina o en un cluster.\\

En este trabajo se implementará una versión simplificada del problema clásico
que consiste en multiplicar dos matrices de NxN.
Lo que se hace es modificar la manera clasica en la que se multiplica las matrices
de manera tal que podamos dividir la misma en sumas y cada una es la entrada de
la función map.
Tenemos que hacer un pre-procesamiento el cual consiste en generar una lista de
tuplas a partir de las dos matrices. Se itera por cada elemento (\code{ai\_ij})
de la matriz A y se guarda en cada tupla el numero de fila del elemento
\code{ai\_ij}, el elemento \code{ai\_ij} y la fila \code{j} de la matriz B.\\

De esta manera, en la funcion map, obtenemos esta lista de tuplas y devolvemos
un par clave, valor donde la clave es la posicion de salida de la matriz
resultante (i, j) y el valor es la multiplicacion del elemento \code{ai\_ij}
contra el elemento de la fila j de la columna B en la columna \code{k}
\code{b\_jk} para k=0, \..., N. \\
Pseudocódigo:

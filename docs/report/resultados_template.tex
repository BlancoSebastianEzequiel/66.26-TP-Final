\paragraph{Salida Amdahl}

    \def\text{Salida de los tiempos en serie y paralelo en milisegundos}
    \def\path{\model_amdahl_output_table.png}
    \def\scale{.6}
    \begin{figure}[!htbp]
    \begin{adjustbox}{addcode={
        \begin{minipage}{\width}}{
            \caption{\text}
        \end{minipage}},rotate=360,center}
        \includegraphics[scale=\scale]{\path}
    \end{adjustbox}
\end{figure}
\FloatBarrier

    De acuerdo a estos datos podemos calcular el speed up maximo, real y teórico.

    \def\text{Speed up real, teorico y maximo segun la cantidad de threads}
    \def\path{\model_amdahl_speed_up_table.png}
    \def\scale{.6}
    \begin{figure}[!htbp]
    \begin{adjustbox}{addcode={
        \begin{minipage}{\width}}{
            \caption{\text}
        \end{minipage}},rotate=360,center}
        \includegraphics[scale=\scale]{\path}
    \end{adjustbox}
\end{figure}
\FloatBarrier

    \def\text{Grafico}
    \def\path{\model_amdahl_speed_up.png}
    \def\scale{.6}
    \begin{figure}[!htbp]
    \begin{adjustbox}{addcode={
        \begin{minipage}{\width}}{
            \caption{\text}
        \end{minipage}},rotate=360,center}
        \includegraphics[scale=\scale]{\path}
    \end{adjustbox}
\end{figure}
\FloatBarrier

    \analisisAmdahl

    \newpage

\paragraph{Salida Gustafson}

    \def\text{Salida de los tiempos en serie y paralelo en milisegundos}
    \def\path{\model_gustafson_output_table.png}
    \def\scale{.6}
    \begin{figure}[!htbp]
    \begin{adjustbox}{addcode={
        \begin{minipage}{\width}}{
            \caption{\text}
        \end{minipage}},rotate=360,center}
        \includegraphics[scale=\scale]{\path}
    \end{adjustbox}
\end{figure}
\FloatBarrier

    \analisisGustafson

    \def\text{Tiempo paralelo y serie en funcion de la dimension de las matrices de entrada}
    \def\path{\model_gustafson_exec_time.png}
    \def\scale{.6}
    \begin{figure}[!htbp]
    \begin{adjustbox}{addcode={
        \begin{minipage}{\width}}{
            \caption{\text}
        \end{minipage}},rotate=360,center}
        \includegraphics[scale=\scale]{\path}
    \end{adjustbox}
\end{figure}
\FloatBarrier

    \clearpage
    \newpage

    Luego a partir de estos datos podemos calcular el speed up y obtuvimos lo
    siguiente:

    \def\text{Tabla de valores del speed up}
    \def\path{\model_gustafson_speed_up_table.png}
    \def\scale{.6}
    \begin{figure}[!htbp]
    \begin{adjustbox}{addcode={
        \begin{minipage}{\width}}{
            \caption{\text}
        \end{minipage}},rotate=360,center}
        \includegraphics[scale=\scale]{\path}
    \end{adjustbox}
\end{figure}
\FloatBarrier

    \def\text{Grafico del speed up}
    \def\path{\model_gustafson_speed_up.png}
    \def\scale{.6}
    \begin{figure}[!htbp]
    \begin{adjustbox}{addcode={
        \begin{minipage}{\width}}{
            \caption{\text}
        \end{minipage}},rotate=360,center}
        \includegraphics[scale=\scale]{\path}
    \end{adjustbox}
\end{figure}
\FloatBarrier

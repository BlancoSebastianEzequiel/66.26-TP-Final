\paragraph{Salida Amdahl}

    \def\text{Salida de los tiempos en serie y paralelo en milisegundos}
    \def\path{\model_amdahl_output_table.png}
    \def\scale{.6}
    \input{report/image.tex}

    De acuerdo a estos datos podemos calcular el speed-up máximo, real y teórico.

    \def\text{Speed-up real, teórico y máximo según la cantidad de threads}
    \def\path{\model_amdahl_speed_up_table.png}
    \def\scale{.6}
    \input{report/image.tex}

    \def\text{Gráfico}
    \def\path{\model_amdahl_speed_up.png}
    \def\scale{.6}
    \input{report/image.tex}

    \analisisAmdahl

    \newpage

\paragraph{Salida Gustafson}

    \def\text{Salida de los tiempos en serie y paralelo en milisegundos}
    \def\path{\model_gustafson_output_table.png}
    \def\scale{.6}
    \input{report/image.tex}

    \analisisGustafson

    \def\text{Tiempo paralelo y serie en función de la dimensión de las matrices de entrada}
    \def\path{\model_gustafson_exec_time.png}
    \def\scale{.6}
    \input{report/image.tex}

    \clearpage
    \newpage

    Luego a partir de estos datos podemos calcular el speed-up y obtuvimos lo
    siguiente:

    \def\text{Tabla de valores de la fracción de la parte secuencial y
    del speed-up escalable}
    \def\path{\model_gustafson_fixed_time_speed_up_table.png}
    \def\scale{.6}
    \input{report/image.tex}

    \def\text{Gráfico del speed-up en función de la parte secuencial.
    Se utilizó la ecuación \code{7}}
    \def\path{\model_gustafson_fixed_time_speed_up.png}
    \def\scale{.6}
    \input{report/image.tex}

    \def\text{Tabla de valores del speed-up}
    \def\path{\model_gustafson_real_speed_up_table.png}
    \def\scale{.6}
    \input{report/image.tex}

    \def\text{Gráfico del speed-up. Se utilizó la ecuación \code{2}}
    \def\path{\model_gustafson_real_speed_up.png}
    \def\scale{.6}
    \input{report/image.tex}

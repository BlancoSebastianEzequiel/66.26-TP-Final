\subsection{Pool}
    El módulo de \code{multiprocessing} puede usar múltiples procesos, pero aún
    tiene que trabajar con el bloqueo global del intérprete de Python, lo que
    significa que no puede compartir memoria entre sus procesos. Por lo tanto,
    cuando intenta iniciar un Pool, necesita copiar variables útiles, procesar
    su cálculo y recuperar el resultado. Esto le cuesta un poco de tiempo para
    cada proceso y lo hace menos efectivo. Pero esto sucede porque se hace
    un cálculo muy pequeño: el \code{multiprocessing} solo es útil para cálculos
    más grandes, cuando la copia de la memoria y la recuperación de resultados
    es más barata (en el tiempo) que el cálculo.\\

    Un cálculo costoso es más eficiente con el \code{multiprocessing}, incluso
    si no siempre se tiene lo que podría esperar (podría tener una aceleración x4,
    pero solo se obtuvo x2). Hay que tener en cuenta que \code{Pool} tiene que
    duplicar cada bit de memoria utilizada en el cálculo, por lo que puede ser
    costoso.\\
    También hay que saber que \code{numpy} tiene una gran cantidad de funciones
    caras escritas en C / Fortran y que ya están en paralelo, por lo que no
    se puede hacer mucho para acelerarlas.
\subsection{Prueba de latecia de pool, python y c}
    A continuación se hizo un calculo el cual requiere iterar unas 10000000 veces
    y hacer una acumulacion para una suma. Por lo tanto se hizo la misma operación
    en c, en python y en python pero usando el modulo \code{pool} donde dividimos el
    trabajo en 4 procesos.
    \def\text{tiempos de la pruebas en segundos}
    \def\path{test_pool.png}
    \def\scale{.6}
    \begin{figure}[!htbp]
    \begin{adjustbox}{addcode={
        \begin{minipage}{\width}}{
            \caption{\text}
        \end{minipage}},rotate=360,center}
        \includegraphics[scale=\scale]{\path}
    \end{adjustbox}
\end{figure}
\FloatBarrier
    Podemos ver que \code{c} es mucho mas eficiente siendo 33 veces mas rápido que
    \code{python}. También podemos ver que usando el módulo \code{pool}, dividiendo
    el trabajo en 4 cores tarda más que el mismo trabajo sin usar \code{pool}
    en python, lo cual demuestra que este módulo es infeficiente como se explicó
    anteriormente debido al uso de recursos.